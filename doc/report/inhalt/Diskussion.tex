\section{Diskussion}

UNTERSCHIEDLICHE KÖRPERHÖHEN AUF LAUFBAND UND LAUFSTRECKE!!!\\
laufband 1,75 (hier Marker gute20 cm VOR der Ebene des BEines!!!)
laufstrecke 1,81 (hier Meterstab GENAU auf der richtigen Höhe)
Mögliche Fehlerquellen:\\
- Markerklebung\\
- Hüftmarker auf Laufband besonders bei langsamen Geschwindigkeiten stark verdeckt\\
- keine statistik (n=1)\\
- kleiner Raum für LAufstrecke, kein uneingeschränktes gehen!!\\
- Marker an verschiedenen Tagen geklebt, daher vllt abweichungen\\

Erst in der Diskussion werden die Daten interpretiert und mit Ergebnissen aus der Literatur verglichen.


Kraftmessungen\\
5 kg fehlen, mögliche Fehlerquellen:\\
geringe Reibung beim Einbauen der Waage, Auftrittsplatte wurde nicht bei Kalibration benutzt\\
Entfernung von Kalibrationsgewichtraum (0 bis 7,75 kg) ist weit entfernt von 75kg!\\

\subsection{Laufband}

geringste Abweichung bei 2 km/h\\
aber angenehmstes Laufen bei 3 km?h\\
diese Abweichung gering und evtl noch kleiner, da Einstellung auf dem Laufband nicht sehr genau in km/h, keine Abstufung vorgenommen.\\
HIER QUELLEN!!\\

\subsection{Vergleiche Band und Strecke}

hier darauf eingehen, dass besseres synchen der trajektorien eine genauere Auswertung erlauben würde!

Die Kräfte in Z-Richtung erlauben Aussagen über die Balance beim Gehen, welche besonders interessant sind für die monopedalen Stützphasen (ehhh, QUELLE?).\\

\subsection{Laufstreckenversuch}
hier keine Glättung der Hüftmarker um zu zeigen, wie stark Verdecken beim Tracking die Werte beeinflusst.\\
Das würde sich durch eine Mittelung über mehrere Doppelschritte auch reduzieren!!\\
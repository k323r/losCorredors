\section{Diskussion}
\subsection{Laufbandversuch}
Am angenehmsten wurde das Gehen bei 3~km~h$^{-1}$ wahrgenommen. Interessanterweise erreichen das Modell des inversen Pendels sowie das des Schwerpendels die höchste Übereinstimmung bei 2~km~h$^{-1}$. Die Übereinstimmung der beiden Modelle ergibt insofern Sinn, als dass unterschiedliche optimale Geschwindigkeiten für Stand- und Schwungbein nicht zum geringsten Energieverbrauch führen würden. Fehler bei der Ermittlung der Periodendauer können ausgeschlossen werden, da die eingestellte Geschwindigkeit mittels Multiplikation der Periodendauer der Schwungphase und Schrittlänge (s.~Tab.~\ref{tab:Erg_Pend}) bestätigt werden konnte. 
Ungenauigkeiten durch Verwendung der anthropometrischen Tabelle sind unwahrscheinlich aufgrund der Überlappung der Ergebnisse der beiden Pendelmodelle, wodurch eine genauere Vermessung der Massenverteilung des Beines die Ergebnisse nicht deutlich verändern sollte.\\
Zusätzlich zu der bestehenden Diskrepanz zwischen Messung und Theorie sei hier der Interpretation des Laufstreckenversuches vorweg gegriffen, bei welchem die angenehme Gehgeschwindigkeit bei 5~km~h$^{-1}$ lag und die als langsam und unangenehm empfundene Geschwindigkeit bei 2~km~h$^{-1}$.\\
Die beiden verwendeten Pendelmodelle scheinen hier an ihre Grenzen zu stoßen und die Komplexität des Gehens nicht vollständig zu erfassen. Dies schränkt die Gültigkeit der Annahme eines konstanten Abstandes von Massenschwerpunkt zu Hüfte von \textcite{witte1992mechanische} ein. \textcite{mcmahon1984muscles} konnte zeigen, dass der Kreisbogen des inversen Pendels durch Rotation und seitliches Abkippen des Beckens sowie durch Knieflexion abgeflacht wird. Das Abkippen der Hüfte kann in der Sagittalebene nicht beobachtet werden und bedürfte eine Untersuchung des Gehens in der Frontalebene. Das Gravitationspendel betrachtet das Bein als ein Pendel mit einem Element, wobei sich das Schwungbein eher als ein Pendel mit drei Elementen exakter beschreiben lässt. Berücksichtigt man diese Aspekte, gewinnt die Modellierung schnell an Komplexität.\\
Es lässt sich zusammenfassen, dass die Modelle des inversen und des Gravitationspendels gute erste Aussagen über das Laufen erlauben, jedoch durch ihre Vereinfachungen auch Fehler eintragen. Es kann mit ihnen die Aussage von \textcite{witte1992mechanische} bestätigt werden, dass das Bein als resonanzschwingunsfähiges System angesehen werden kann. Dass der geringste Arbeitsaufwand für Gehen zwischen 4~km~h$^{-1}$ \parencite{cavagna1976sources} und 4,5~km~h$^{-1}$ \parencite{cavagna2000role} liegt, weicht stark von der als angenehm wahrgenommenen Geschwindigkeit auf dem Laufband ab, deckt sich jedoch gut mit der als angenehm wahrgenommenen Geschwindigkeit auf der Laufstrecke (5~km~h$^{-1}$). Die starke Abweichung der als angenehm wahrgenommenen Geschwindigkeit kann mit der sehr geringen Zeit zum Einlaufen auf dem Laufband zusammenhängen. Eine längerer Eingewöhnungsphase bei jeder Geschwindigkeit könnte die Wahrnehmung beeinflussen.

\subsection{Laufstreckenversuch}
Zur Überprüfung der Ergebnisse werden die Bodenreaktionskräfte zunächst untersucht. Bei der angenehmen Geschwindigkeit von 5~km~h$^{-1}$ liegen die Maxima bei 103\% und 109\% und das Minimum bei 70\%. Für 5~km~h$^{-1}$ stimmen die Werte gut mit den von \textcite{perry2010gait} ermittelten Durchschnittswerten von je 110\% für die Maxima und 80\% für das Minimum annähernd überein. Es ist jedoch zu erkennen, dass alle prozentualen Werte unter denen von Perry liegen. Dies deckt sich mit der Tatsache, dass das Körpergewicht zu keinem Moment bei 2~km~h$^{-1}$ vollständig vom Standbein getragen wird, wie es eigentlich der Fall sein sollte. Mögliche Ursachen für die zu niedrigen Gewichtswerte scheinen in der Datenaufnahme oder in der Kalibrierung zu liegen, da sich der Gewichtsunterschied auch bei anderen Kommilitonen, welche an anderen Tagen gemessen haben, ausbildete. Eine mögliche Ursache ist in der Kalibrierung zu finden, welche mit 0~kg, 1,5~kg, 3,5~kg und 7,75~kg durchgeführt wurde. Die Toleranz der verwendeten Personenwaage könnte sich bei Gewichten in einer Größenordnung darüber deutlich auswirken. So könnten die 5~kg Abweichung schon durch 0.5~kg Waagentoleranz entstehen. Auch ist zu berücksichtigen, dass während der Kalibrierung der vertikalen Waagenachse die Fußplatte nicht montiert war, welche in Abbildung~\ref{fig:setup_combo}~B dargestellt ist. Diese könnte ebenfalls Einfluss auf die statischen Kräfte durch Gewichtskraft sowie dynamischen Kräfte durch Trägheitskraft genommen haben.\\
Bei allen drei Geschwindigkeiten ist der Übergang zwischen initialem Kontakt und Stoßdämpfungsphase deutlich zu erkennen, gekennzeichnet durch das kurzzeitige Einbrechen des Belastungsanstieges. Diese Stufe ist am Deutlichsten bei 7~km~h$^{-1}$ zu erkennen, wo es zu einem klar abgegrenzten Plateau kommt. Dieses Plateau kommt durch die Flexion des Knies zustande (s.~Abb.~\ref{fig:comp_knee_angle}) und dient dem Abfangen der Belastung bei Bodenkontakt \parencite{perry2010gait}. Da mit steigender Geschwindigkeit der dynamische Anteil der Belastung immer höher wird, steigen auch die Maximalbelastungen in der Stoßdämpfungsphase. Diese Maxima wiederum kor-relieren deutlich mit der maximalen Knieflexion, was sich mit den Ergebnissen von \textcite{kirtley1985influence} deckt. Der Duty-Faktor von 0.55 bei 7~km~h$^{-1}$ zeigt hier, dass der Proband sich kurz vor dem Übergang zwischen Gehen und Laufen befindet.\\
Die Momente im Knie unterstützen diese Ergebnisse. Der positive Momentverlauf während der Stoßdämpfungsphase und das negative Maximum in der terminalen Standphase decken sich mit den Daten von \textcite{perry2010gait}. Während der Stoßdämpfungsphase kommt es zu einem positiven Moment im Knie, welches zu einer Flexion führt, wie im Verlauf des Kniewinkels deutlich zu erkennen. Zum Ende der Standphase tritt ein negatives Moment auf, was eine Extension bewirken würde. Trotz des negativen Moments kommt es zu einer noch stärkeren Flexion als in der Stoßdämpfungsphase. Dies zeigt, dass die inverse Kinetik nicht die von den Muskeln aufgebrachten Kräfte berücksichtigt und daher die Bewegung nicht ausschließlich erklären kann.\\
\textcite{perry2010gait} beschreiben zusätzlich ein weiteres positives Moment, welches die Extension des Knies in der Schwungphase einleitet. Dieses konnte hier nicht beobachtet werden, da mit dem Lösen des Fußes vom Boden keine Bodenreaktionskräfte mehr aufgenommen wurden.\\
Interessant ist hierbei die Entwicklung des Moments im Knie über die drei Geschwindigkeiten unter Berücksichtigung der Knieflexion \ref{fig:comp_knee_mom}. Betrachtet man das Gehen bei 2~km~h$^{-1}$, so ist nur eine geringe Flexion von 22$^{\circ}$ des Knies zu erkennen. Dieses Ergebnis passt zu dem sehr geringen Moment (s.~Abb.~\ref{fig:comp_knee_mom}). Die maximale Flexion in der Stoßdämpfungsphase hängt deutlich mit der Geschwindigkeit zusammen. Interessant ist hier, dass sich der Flexionswinkel von 2 auf 5~km~h$^{-1}$  viel stärker verändert als von 5 auf 7~km~h$^{-1}$, das Moment im Knie jedoch zwischen den ersten beiden Geschwindigkeiten nur gering ansteigt und zwischen 5 und 7~km~h$^{-1}$ deutlich größer wird. Es wird vermutet, dass dem positiven Moment bei 5 und 7~km~h$^{-1}$ eine muskuläre Kraft entgegengesetzt wird, welche eine übermäßige Flexion des Knies in der Stoßdämpfungsphase verhindert. Das negative Moment am Ende der terminalen Standphase steigt von 2 zu 5~km~h$^{-1}$ ebenfalls an, ist jedoch bei 7~km~h$^{-1}$ kaum zu erkennen. Obwohl die Knieflexion bei 7~km~h$^{-1}$ mit 72$^{\circ}$ am Stärksten ausgeprägt ist, scheint hierfür am wenigsten Kraft notwendig zu sein. Eine so starke Veränderung des Verlaufs des Moments kann in der Literatur nicht beobachtet werden, was auf die nur einmalige Durchführung des Versuches zurückgeführt wird.\\
Nicht mit den Ergebnissen von Perry übereinstimmend ist das Moment im Knöchel, welches durchgängig positiv ist. \textcite{perry2010gait}, \textcite{eng1995kinetic} und \textcite{lay2006effects} beschreiben im Knöchel ein leicht negatives Moment während der Belastungsantwort aufgrund der Plantarflexion während des Aufsetzens des Fußes. Während der mittleren und terminalen Standphase steigt das Moment dann an und ähnelt den hier gemessenen Werten. Das Maximum des Moments in der terminalen Standphase mit einem Abfallen in der Vorschwungphase kann auch in den Ergebnissen dieser Arbeit beobachtet werden. Aufgrund der einmaligen Durchführung der Versuche wird auf eine Interpretation dieser Abweichung verzichtet, da nicht gesagt werden kann, ob es zum Gangmuster des Probanden gehört.\\
Für das Moment im Hüftgelenk unterscheiden sich die Ergebnisse in der Literatur, jedoch ist bei allen zu Beginn der Standphase ein positives Moment zu beobachten, gefolgt von einem negativen Moment in der terminalen Standphase. Dieses Ergebnis deckt sich mit den hier erhobenen Daten. Auffällig bei den Ergebnissen der X- und Y-Kräfte in der Hüfte sind die starken Ausschläge am Anfang der Standphase. Diese Schwankungen werden dadurch erklärt, dass zu diesem Zeitraum der Hüftmarker von der Hand verdeckt war und dessen Position abgeschätzt wurde. Hier würde eine größere Anzahl an ausgewerteten Doppelschritten sich sehr stark bemerkbar machen, da solche Artefakte durch die Mittlung stark abgeschwächt würden.


\subsection{Vergleich Laufband und Laufstrecke}
Ein Vergleich der Handtrajektorien bei 2~km~h$^{-1}$ erlaubt keine schlüssige Aussage. Dies kann zum einen an der fehlenden Mittlung über mehrere Doppelschritte liegen, zum anderen daran, dass bei so geringen Geschwindigkeiten die Arme noch keine Ausgleichsbewegungen gegen die Beinbewegung durchführen müssen und die Stabilität des Gehens noch keine dynamische ist, sondern eher ein Balancieren auf der Stelle und damit mehr einer statischen Stabilität gleicht \parencite{barbareschi2015statically}.\\
Bei 5 und 7~km~h$^{-1}$ ist ein deutlicher Unterschied in den Handtrajektorien zwischen den Versuchen zu erkennen. Die stärkere Auslenkung in X- und Y-Richtung bei 5 und 7~km~h$^{-1}$ auf dem Laufband ist interessant unter der Annahme der Arme als passive Masse-Dämpfer nach \textcite{pontzer2009control}. Demnach dienen die Arme der Reduktion der Rotation des Oberkörpers um die Längsachse. Die stärkere Auslenkung auf dem Laufband würde demnach bedeuten, dass hier der Oberkörper stärker zur Rotation neigt, was jedoch durch die stärkere Armbewegung ausgeglichen wird und daher der Energieverbrauch nicht beeinträchtigt sein sollte \parencite{pontzer2009control}. Während andere Arbeitsgruppen deutliche Unterschiede zwischen Laufband und Laufstrecke für Hüftbewegung in der Frontalebene gefunden haben \parencite{alton1998kinematic}, konnten keine derartigen Unterschied für die Armbewegung gefunden werden. Jedoch werden die Ergebnisse dieser Untersuchung wie bei \textcite{alton1998kinematic} auf das ungewohnte Gehen auf dem Laufband zurückgeführt. Das stärkere Schwingen der Arme könnte eine Ausgleichsbewegung für instabiles Gehen auf dem Laufband sein.\\
Die Ergebnisse zu der Körperachse zeigen interessanterweise, dass auf dem Laufband der Körper stärker nach vorne geneigt ist als auf der Laufstrecke. Dies widerspricht den erwarteten Ergebnissen. Aufgrund der tatsächlichen Positionsänderung auf der Laufstrecke wurde erwartet, dass hier die kinetische Energie zur Vorwärtsbewegung aus dem vorgebeugten Oberkörper gewonnen wird. Besonders die Ergebnisse bei 5 und 7~km~h$^{-1}$ zeigen das genaue Gegenteil. Es sei darauf hingewiesen, dass aufgrund der fehlenden Mittlung über mehrere Durchläufe es sich hier auch um Ausreißer handeln könnte. So wurde im Nachhinein festgestellt, dass bei 2~km~h$^{-1}$ auf dem Laufband der Blick nach unten gerichtet war, was großen Einfluss auf den Winkel der Körperachse hat. Dieses Ergebnis wird daher kritisch betrachtet und der stärker nach vorne gebeugte Oberkörper auf dem Laufband auch hier auf die ungewohnte Laufsituation zurückgeführt. Eine mögliche Erklärung ist die beobachtete Wahrnehmung, dass auf dem Laufband die Gefahr bestand, hinten über zu kippen. Dieses Gefühl würde eine unbewusste Haltungsänderung in eine vorgebeugte Lage erklären, welche als sicherer wahrgenommen wird. Für das Gehen auf der Laufstrecke sei noch erwähnt, dass aufgrund der geringen Abmessungen des Raumes nach Passieren des Messbereiches sehr schnell abgebremst werden musste. Dieses Abbremsen könnte ebenfalls mit einem zurückgelehnten Oberkörper in Zusammenhang stehen.
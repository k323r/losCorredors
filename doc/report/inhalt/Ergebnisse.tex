\section{Ergebnisse}
\subsection{Laufband}
Hier werden nur die Ergebnisse, durch Graphen und Bilder aufgezeigt und deren Inhalt beschrieben. Es werden noch kein Rückschlüsse und Interpretationen gezogen.\\

Die Laufbandversuche werden in \autoref{Fadenpendel} mit einem theoretischen Fadenpendel verglichen.\\
\begin{table}[h!]
\centering
\caption[Vergleich Fadenpendel]{Hier könnte ihre Werbung stehen! Tel: 0800-LATEX-WERBUNG}
\label{Fadenpendel}
\begin{tabular}{C{3.5cm}C{3cm}C{3cm}C{4cm}}
\textbf{Geschwindigkeit $[\,km\cdot h^{-1}\,]$} & \textbf{subjektive \newline Einschätzung} & \textbf{berechnete \newline Frequenz} & \textbf{Abweichung vom \newline Fadenpendel $[\,\%\,]$}\\
\midrule[1.5pt]
1 & total anstrengend & ? & XX\\
2 & joar & ? & XX\\
3 & wow & ? & XX\\
4 & wooow & ? & XX\\
5 & nääh & ? & XX\\
6 & miau & ? & XX\\
7 & muh & ? & XX\\
\bottomrule[1.5pt]
\end{tabular}
\end{table}
\subsection{Laufband (Fadenpendel)}
\subsection{Laufstrecke}
\subsection{Vergleich Laufband und Laufstrecke}
\textbf{Vergleich Laufstrecke und Laufband}\\
Mithilfe von Winkelmessungen des Oberkörpers, Arme und Beine können die beiden Gangarten verglichen und analysiert werden\\
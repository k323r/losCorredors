\paragraph*{Fazit}
Inverse kinematics, the estimation of joint kinematics based on measured trajectories of skin-mounted markers, is complicated by instrumental errors and soft tissue artefacts \cite{groote2008kalman}\\

Fehlender Statistik macht es schwer, Aussagen zu beurteilen, welche von der Literatur abweichen, da hier nicht sichergegangen werden kann, dass es sich um tatsächliches gangmuster oder einen Ausreißer handelt. 

Beide untersuchten Pendel-Modelle erreichten eine relativ gute Übereinstimmung mit der als angenehm wahrgenommenen Geschwindigkeit. Jedoch scheinen die starken Vereinfachungen, wie z.B. das masselose Bein im inversen Pendel Modell und die fehlende Drei-Segment-Betrachtung des Beines bei dem Gravitationspendel Fehler einzutragen, die die Abweichungen erklären könnten. Die Betrachtung des Beines mit kompelexeren Modellen könnte hier eine bessere Abbildung des Systems Bein ermöglichen.\\
Der Vergleich der Armbewegung scheint auf stärkere Momente um die Längsachse auf dem Laufband als auf der Laufstrecke hinzuweisen. Die Hypothese, dass auf der Laufstrecke die tatsächliche Ortsänderung durch ein Vorbeugen des Körpers erreicht wird, konnte nicht bestätigt werden. Interessanterweise ist gerade auf der Laufstrecke der Körper weiter zurückgelehnt. Ungewohntes Gehen auf dem Laufband und ein verfrühtes Abbremsen auf der Laufstrecke wurden hier als einflussreiche Faktoren gesehen. Es wird vermutet, dass eine erneute Durchführung der Versuche mit mehr Eingewöhnung und einer längeren Laufstrecke großen Einfluss auf die Ergebnisse hat und dadurch zu anderen Ergebnissen führen könnte.\\
Die kinetische Untersuchung zeigt große Übereinstimmung mit der Literatur und scheint am wenigsten von der einmaligen Durchführung der Versuche beeinflusst zu sein. Die inverse Kinetik dagegen zeigt starke Abweichungen in einigen Gelenken für den Momenten-Verlauf. Leider ist eine Interpretation dieser Unterschiede nicht möglich, da nicht sichergestellt werden kann, dass es keine unnatürlichen Ereignisse sind, die zu diesen Ergebnissen führen.

\paragraph*{Ausblick}
Die qualitative Untersuchung des menschlichen Ganges eines Probanden erlaubt trotz der einmaligen Durchführung der Versuche interessante Schlüsse. So zeigen die Daten zu den Trajektorien Potential für die Übertragung auf bipedale Robotersysteme. In Kombination mit Lernalgorithmen könnte so das maschinelle Lernen eines Gangmusters enorm beschleunigt werden. Der zeitlich Verlauf der Kräfte und Momente ist essentiell für die Steuerung und Regelung eines solchen Systems. Dies erlaubt eine angepasste Auslegung der Motoren. Hier würden sich weitere Untersuchungen zu muskulären Aktivierung in den verschiedenen Gangphasen anbieten. Diese würden zusammen mit Kräften und Momenten ein Planen des Reaktionen der Steuerung auf verschiedene Lastfälle erlauben. Anstatt ein reagierendes System zu bauen könnte so ein agierendes System geschaffen werden, was nur noch auf Sonderfälle reagiert.\\
Ebenfalls interessant ist das Verständnis des menschlichen Ganges für die Verbesserung von Exoskeletten. \textcite{barbareschi2015statically} beschreiben anschaulich die Verwendung in der Rehabilitation. Hier kann zum Einen durch statische Stabilität die Sicherheit des Patienten zu jedem Zeitpunkt gesichert werden und gleichzeitig durch ein möglichst natürliches Gangmuster des Exoskeletts der Eindruck eines uneingeschränkten Gehens entstehen.

\section{TODO}
\subsection{Geschw. und Beschl Linear und Winkel}
- pur plotten\\
- angucken\\
- sinnvoll plotten\\
\subsection{Scilab}
- Einlesen Tabellen\\
- Massenschwerpunkte bestimmen\\
- Geschwindigkeiten und Beschleunigung (linear und winkel)\\
- Daten glätten, gleitender Mittelwert\\
- plotten\\
-----------------------------------------\\
- Kalibrierung der Waage\\
- Messdaten bereinigen (Drift und Nullmessung)\\
- Messdaten in Kräfte umrechnen\\
(y-Richtung Waage = x-Richtung der Videos)\\
(z-Richtung Waage = y-Richtung der Videos)\\
- Bestimmung des genauen Ortes der Bodenreaktionskraft\\
- Berechnung der inversen Dynamik\\
- Zeitliche Syncronisation der Datensätze (Startbild und Aufnahmefrequenz)\\
\clearpage

\section{Checkliste Inhalt}
Laufband\\
Vergleich mit Fadenpendel -> Eigenfrequenz\\
aufzubringende Energie bei versch. Geschw. mit Pendel vergleichen\\
Vergleich quantifizieren ( z.B. 1 -10)\\
-----------------------------------------
Datenauswertung hierzu steht!
-----------------------------------------\\\\

Laufband mit Laufstrecke\\
Unterschiede zwischen Laufen auf einem Fleck (Laufband) und tatsächlicher Ortsänderung(Laufstrecke)\\
sieht man Unterschied, wenn Kraft in x-Richtung aufgebracht wird?\\
Hypothese: Durch Vorbeugen bei Orständ. Aufbringen zustätzl. Energie\\
auf Laufband: Laufband stellt Energie bereit\\
------------------------------------------------------\\
Laufstrecke\\
Auswertung durch inverse Kinematik (Winter 2009)\\
Kräfte und Momente für alle Gelenke analysieren und interpretieren\\
Bedeutung der Daten hinsichtlich bionischer oder medizintechnischer Anwendungen\\
weitere Schlussfolgerungen (s. Winter) und mögliche weiterführende Berechnunge/Untersuchungen\\
------------------------------------------------------\\
Anwendung auf andere Fortbewegungssysteme/ mögl. Anwendungen\\
Robotik, 4/6/8 Beine\\
Exoskelette Programmierung für natürlich Unterstützung des Menschen\\
------------------------------------------------------\\
auf Fehlen der Statistik eingehen\\
kurz beschreiben, welche Daten notwendig wären, welche Verfahren geeignet wären\\
------------------------------------------------------\\
\textbf{Weiterführende Literatur!!}\\
\clearpage


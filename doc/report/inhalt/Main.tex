\section{TODO}

\textbf{Kirtley et al 1985}

knee angle and moment and changes over walking speed\\
peak knee reflexion strongly correlated wit walkin gsped\\

discussion;\\
strong correlation of cadence, stride length and velocity\\
velocity highest correlation with stance phase\\
cadence highest correlation with swing phase knee flexion

GOOD ARGUMENTATION!\\
Cadence, stride length and velocity\\
Eigenfrequency cant be changed, therefore energy is needed for decelaration and acceleration when walking faster or slower\\
walking becomes \textbf{more difficult}\\

shortening stance phase due to the fact that swing phase cant be shortened as easily as stance phase (gibt quelle 11 an, nachgucken! zitieren!)

no change in knee extension peaks, kurz vor der landung,bei 2/3 der standphase\\

Kniewinkel\\
Standphase: extension kaum verändert, flexionsmaximum steigt\\
Schuwngphase: kein große Änderung des lfexionswinkels??\\

\textbf{Kuo2007}
check cincluiso Seite 35\\
Pendulum ist ne gute theorie! aber irgendwas mit fliegender Kugel und "dnymic walking"

\textbf{whittle1996}\\
Seite 8 und 10\\
gute graphen!\\

\textbf{READ!!!}\\
Alton 1998\\
Jordan 2007\\


\subsection{MundM}
Beschriftung Abbildungsteile laufsteg und laufband
\subsection{Geschw. und Beschl Linear und Winkel}
- pur plotten\\
- angucken\\
- sinnvoll plotten\\
\subsection{Scilab}
- Einlesen Tabellen\\
- Massenschwerpunkte bestimmen\\
- Geschwindigkeiten und Beschleunigung (linear und winkel)\\
- Daten glätten, gleitender Mittelwert\\
- plotten\\
-----------------------------------------\\
- Kalibrierung der Waage\\
- Messdaten bereinigen (Drift und Nullmessung)\\
- Messdaten in Kräfte umrechnen\\
(y-Richtung Waage = x-Richtung der Videos)\\
(z-Richtung Waage = y-Richtung der Videos)\\
- Bestimmung des genauen Ortes der Bodenreaktionskraft\\
- Berechnung der inversen Dynamik\\
- Zeitliche Syncronisation der Datensätze (Startbild und Aufnahmefrequenz)\\
\clearpage

\section{Checkliste Inhalt}
Laufband\\
Vergleich mit Fadenpendel -> Eigenfrequenz\\
aufzubringende Energie bei versch. Geschw. mit Pendel vergleichen\\
Vergleich quantifizieren ( z.B. 1 -10)\\
-----------------------------------------
Datenauswertung hierzu steht!
-----------------------------------------\\\\

Laufband mit Laufstrecke\\
Unterschiede zwischen Laufen auf einem Fleck (Laufband) und tatsächlicher Ortsänderung(Laufstrecke)\\
sieht man Unterschied, wenn Kraft in x-Richtung aufgebracht wird?\\
Hypothese: Durch Vorbeugen bei Orständ. Aufbringen zustätzl. Energie\\
auf Laufband: Laufband stellt Energie bereit\\
------------------------------------------------------\\
Laufstrecke\\
Auswertung durch inverse Kinematik (Winter 2009)\\
Kräfte und Momente für alle Gelenke analysieren und interpretieren\\
Bedeutung der Daten hinsichtlich bionischer oder medizintechnischer Anwendungen\\
weitere Schlussfolgerungen (s. Winter) und mögliche weiterführende Berechnunge/Untersuchungen\\
------------------------------------------------------\\
Anwendung auf andere Fortbewegungssysteme/ mögl. Anwendungen\\
Robotik, 4/6/8 Beine\\
Exoskelette Programmierung für natürlich Unterstützung des Menschen\\
------------------------------------------------------\\
auf Fehlen der Statistik eingehen\\
kurz beschreiben, welche Daten notwendig wären, welche Verfahren geeignet wären\\
------------------------------------------------------\\
\textbf{Weiterführende Literatur!!}\\
\clearpage


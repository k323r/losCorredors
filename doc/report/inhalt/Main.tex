\section{Notes}

\textbf{Kirtley et al 1985}

knee angle and moment and changes over walking speed\\
peak knee reflexion strongly correlated wit walkin gsped\\

discussion;\\
strong correlation of cadence, stride length and velocity\\
velocity highest correlation with stance phase\\
cadence highest correlation with swing phase knee flexion

GOOD ARGUMENTATION!\\
Cadence, stride length and velocity\\
Eigenfrequency cant be changed, therefore energy is needed for decelaration and acceleration when walking faster or slower\\
walking becomes \textbf{more difficult}\\

shortening stance phase due to the fact that swing phase cant be shortened as easily as stance phase (gibt quelle 11 an, nachgucken! zitieren!)

no change in knee extension peaks, kurz vor der landung,bei 2/3 der standphase\\

Kniewinkel\\
Standphase: extension kaum verändert, flexionsmaximum steigt\\
Schuwngphase: kein große Änderung des lfexionswinkels??\\

\textbf{Kuo2007}
check cincluiso Seite 35\\
Pendulum ist ne gute theorie! aber irgendwas mit fliegender Kugel und "dnymic walking"

\textbf{whittle1996}\\
Seite 8 und 10\\
gute graphen!\\

\textbf{Danion2003}\\
Gangzyklen sind stark Variabilität unterworfen\\

\textbf{Masaad-etal2007}\\
introduciotn:\\
flat walking muscles work less efficient\\
bouncy walking muscles work more but also more efficient!\\
compare walking model types on a meta-level!!!!!!!!!!!!!!!!!!!!!\\

\textbf{alexander1992}\\
introduction: why we cant walk faster!!!\\

\textbf{READ!!!}\\
Alton 1998\\
Jordan 2007\\

\section{Checkliste Inhalt}
\textbf{Bewertungskriterien}\\
Vorgehensweise:\\
welche Auswertungen wurden durchgeführt\\
wie viel Hingabe liegt in der Erstellung\\
wurde alles ausgewertet?\\
------------------------------------------------------\\
\textbf{Laufband mit Laufstrecke vergleichen}\\
dazu Litertur suchen -> Vorwärtsbewegung ist kontrolliertes Fallen (KUO 2007)\\
Treibende Kraft aus Gravitationskraft und Oberkörperneigungswinkel berechnen\\
hier müsste eigentlich rauskommen, dass keine Kraft sich ergibt über einen Schrittzyklus, da das eine Pendelbewegung ist\\
------------------------------------------------------\\
\textbf{Laufstrecke}\\
Auswertung durch inverse Kinematik (Winter 2009)\\
Kräfte und Momente für alle Gelenke analysieren und interpretieren\\
Bedeutung der Daten hinsichtlich bionischer oder medizintechnischer Anwendungen\\
weitere Schlussfolgerungen (s. Winter) und mögliche weiterführende Berechnunge/Untersuchungen\\
auftretende Kräfte und Momente angucken und interpretieren\\
was kann man daraus ablesen\\
welches Kraft/moment tritt wann auf, warum ist das so?\\

bei welcher Gangphase passiert was, was kann daraus gezogen werden?\\
siehe Winter: was kann man noch weiter berechnen, Ansatz der Hebelarme etc...\\
welche Auswirkung hat dieses Wissen für technische Anwendung, zum Beispiel die ideale Dämpfung\\
Robotik, was kann man für bipedales Gehen für Gangmuster (central pattern generators - CPG) aus den Untersuchungen ziehen -> Robotik, Medizintechnik, Exoskelette\\
------------------------------------------------------\\
Anwendung auf andere Fortbewegungssysteme/ mögl. Anwendungen\\
Robotik, 4/6/8 Beine\\
Exoskelette Programmierung für natürlich Unterstützung des Menschen\\
------------------------------------------------------\\
auf Fehlen der Statistik eingehen\\
kurz beschreiben, welche Daten notwendig wären, welche Verfahren geeignet wären\\
------------------------------------------------------\\


\section{TODO}


TODO MITTWOCH:\\
Einleitung überarbeiten\\
Neue Auswertung Kraftkram (siehe Perry und Paper)\\
BRK so neu plotten, dass 2 kmh am Längsten ist, da Anfangs- und Endpunkt genau treffen\\
Anhang bauen\\

TODO DONNERSTAG:\\
Ergebnisse zu allen bildern Texte!\\
Abbildung Perry überarbeiten\\
Diskussion schreiben\\


\section{Material und Methoden}
In dem Teil der Material \& Methode wird ausschließlich die Vorgehensweise während der Datenerhebung und Auswertung beschrieben. Dazu gehören alle mathematischen Formeln, die genutzt wurden, um die Ergebnisse zu berechnen. Es werden sämtlich Werkzeuge beschrieben und Versuchsaufbauten rekonstruierbar dargestellt.

\subsection{Proband}
Der Proband, dessen Daten in den durchgeführten Laufversuchen aufgenommen wurden, ist männlich, 181 cm groß und wiegt 75 kg.
\subsection{Laufband}
\subsubsection{Versuchsaufbau}
Kern des Versuchsaufbaus für die Laufbandversuche ist ein Laufband
\subsubsection{Datenaufbereitung}
Gleichungen angeben\\
Daten filtern!\\
\subsection{Laufstrecke}
\subsubsection{Versuchsaufbau}
\subsubsection{Datenauswertung}
\textit{Gleichungen angeben}\\
\textbf{Kalibrierung}\\
- Nullmessung zur Bestimmung des Waagendrifts\\
- Kalibrierungsmessung\\
--> Bestimmung der der Abhängigkeit zwischen Belastungskraft und Messdaten\\
- Reduktion des Waagendrifts\\
- Regressionsgleichung von Messdaten und Belastungskraft bestimmen\\
\textbf{Datenaufnahme}\\
- Kräfte zusammenfassen\\
- Waagendrift aus Rohdaten rausrechnen\\
- Mittels der Kalibrierungsgleichungen die Rohdaten in Kräfte umrechnen\\
\textbf{Skalierung und Synchronisation!!}\\
- Bei der Skalierung wird die Datenrate der Videorate angepasst (hier also nur jeder zweite Datensatz). Gegebenenfalls muss zwischen den Datensätzen interpoliert werden.\\
- Bei der Datensynchronisation findet ein Abgleich des Videomaterials und der Bodenreaktionskräfte statt\\
\textbf{Digitalisierung des Videomaterials}\\
- Skalieren der Videoaufnahme (ACHTUNG! Referenzbild mit Maßstab erforderlich?!?!?)\\
- Tracken von allen Gelenken\\
- Segmentschwerpunkte berechnen (Fuß, Unter- und Oberschenkel)\\
\textbf{Datenberechnung\ KINEMATIK??}\\
- Berechnung von linearen Geschwindigkeiten und Beschleunigungen\\
- Berechnung von Winkelgeschwindigkeiten und Winkelbeschleunigungen\\
\textbf{Datenfilterung (gleitender Mittelwert)}\\
- ACHTUNG! Je nach Anzahl von Stützstellen und Iterationen müssen Bilder vor und nach dem Schrittzyklus in die Digitalisierung einbezogen werden. z.B. 3 Stützstellen und eine Iteration benötigt 1 Bild vorher und ein Bild nachher, um i-1 und n+1 zu berücksichtigen.\\
\textbf{Kinetische Berechnungen (Wagenzentrum rausrechnen?)}\\
Auf der Grundlage von David A. Winter werden:\\
- Kräfte und Momente in den Gelenken berechnet\\
- Berechnung mittels inverser Dynamik\\
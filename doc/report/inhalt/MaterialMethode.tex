\section{Material und Methoden}

\subsection{Proband}
Alle hier dargestellten Daten beziehen sich auf eine männliche Person mit einer Körpergröße l~=~181 cm und einem Gewicht von m~=~75~kg. Zur genauen Erkennung der Gelenke in beiden Experimenten werden die Gelenke Hals, Schulter, Ellenbogen, Handgelenk, Hüfte, Knie, Knöchel, Ferse und Ballen mit reflektierenden Markern versehen.

\subsection{Material}
HIER: TABELLE MIT KAMERA, SCHEINWERFERN, LAUFBAND, WAAGE ETC.\\
DANN NUR NOCH BILDER ZEIGEN FÜR DEN AUFBAU\\
AUSRICHTUNG DER KAMERA GENERELL ERKLÄREN
\subsubsection{Laufband}
Der Aufbau besteht aus einem mercury 4.0 Laufband (h/p/cosmos sports \& medical GmbH, Nussdorf-Traunstein, Deutschland), einer Samsung VP-HMX20C Videokamera (Samsung AG Seoul, Südkorea) und zwei weißen 500W Baustrahlern (MARKE???). Abbildung~\ref{fig:laufbnd_stp} zeigt den Aufbau mit der Kamera 5~m vom Laufband entfernt. Die Bildebene ist parallel zur Sagittalebene ausgerichtet.

\begin{figure}[h!]
	\centering
	\includegraphics[width=0.5\linewidth]{bilder/mat_met/Laufband_setup}
	\caption[Aufbau Laufband Versuch]{Aufbau des Laufband-Versuches mit Laufband LB, Kamera K und Scheinwerfern SW Abstand zur Kamera $A_{Kam}~=~5~m$ }
	\label{fig:laufbnd_stp}
\end{figure}

\subsubsection{Laufstrecke}
Dieselbe Videokamera und Baustrahler wie bei dem Laufband-Versuchen werden verwendet. Der Proband läuft in diesem Experiment über einen Laufstrecke (Eigenbau Hochschule Bremen, Deutschland). In den Steg ist ein Quarzkristall-3-Komponenten-Dynanometer Typ 9257B (Kistler Gruppe Winterthur, Schweiz) zum Messen der Kräfte eingebaut, welches im Folgenden als Waage bezeichnet wird. Die Waage ist über einen Mehrkanal-Ladungsverstärker Typ 5070A (Kistler) mit einem Computer verbunden. Abbildung~\ref{fig:laufstg_stp} (nicht maßstäblich) zeigt den Aufbau mit der Kamera 5~m vom Laufstrecke entfernt. Die Kamera steht zentriert vor der Waage und die Bildebene ist parallel zur Saggitalebene ausgerichtet.
\begin{figure}[h!]
	\centering
	\includegraphics[width=0.7\linewidth]{bilder/mat_met/Laufstrecke_setup}
	\caption[Aufbau Laufstrecke Versuch]{Aufbau des Laufstrecke-Versuches mit Laufstrecke LS, Kamera K, Scheinwerfern SW, Computer C, Waage W und Ladungsverstärker LV. Länge Steg $L_S~=~6~m$ und Abstand zur Kamera $A_{Kam}~=~5~m$}
	\label{fig:laufstg_stp}
\end{figure}

\subsection{Methoden}
\subsubsection{Laufband}
Ausgehend von 1~km/h werden in 1~km/h-Schritten sieben Geschwindigkeiten untesucht. Das Gehen wird subjektiv von 1 (angenehm) bis 10 (unangenehm) bewertet.Je zwei Gangzyklen werden mit einer Bildrate von 50~Hz, einer Belichtung von 1/1000~s und auf 5~m fixierten Fokus aufgenommen.\\
Die Videos werden anschließend mit dem Programm ffmpeg 2.1 (LGP License, ffmpeg.org) in Einzelbilder zerlegt. In ImageJ (National Institutes of Health Bethesda, Maryland) werden mithilfe des Plugins MTrackJ \textbf{CITE(Meijering 20120)} die x- und y-Koordinaten aller Gelenke digitalisiert und im .mdf-Format gespeichert.

\subsubsection{Laufstrecke}
EINSTELLUNGEN LADUNGSVERSTÄRKER???\\
Die Kameraeinstellungen vom Laufbandversuch übernommen finden auch hier Anwendung. Die Digitalisierung der Koordinaten ist wie oben beschrieben durchgeführt worden. Die Waagensignale werden mit 100 Hz und 200~N/V vom Ladungsverstärker an den Computer weitergeleitet. Mit DASYLab (Measurement Computing, Norton, USA) werden die Eingangssignale verarbeitet und alle 8 Kanäle im ASCII-Format gespeichert.
Für eine 4-Punkt-Kalibration wird die Waage in alle drei Raumrichtungen mit 0, 1, 3,6 und 7,75~kg belastet. Der Waagendrift wird über 60~s ohne Belastung für jede Raumrichtung ermittelt.\\
Blick geradeaus, um nicht auf den Gang nicht an die Waage anzupassen


\subsection{Datenauswertung mit Scilab}
In diesem Abschnitt wird auf die grundlegenden Gleichungen eingegangen, welche in Scilab (Scilab Enterprises S.A.S., Orsay Cedex, France) verwendet wurden, um die Rohdaten auszuwerten. Für die exakte Implementierung sei auf die beigefügte CD mit dem Quellcode verwiesen.\\
\\
HIER NOTWENDIG ZENTRALDIFFERENZ ETC HINZUSCHREIBEN??\\

Für die Auswertung des Laufbandes wird die Periodendauer von Beginn der initialen bis Ende der terminalen Schwungphase ermittelt. Mit der Beinlänge als Länge eines idealen Pendels wird die Eigenfrequenz bestimmt.\\
Zum Vergleich der beiden Versuche werden der Winkel des Oberkörpers (Linie zwischen Nacken- und Hüft-Marker) zur Senkrechten sowie die Trajektorie der Hand herangezogen. Ersteres wird WIE GEMACHT?!?!?\\
Für die Handtrajektorie wird von den X-Koordinaten der Laufstrecke ein $\delta x = Geschw \delta t$ abgezogen, um diese mit den Laufbandtrajektorien vergleichen zu können.\\
STIMMT DAS??\\
FELIX WIE HAST DU DIE GESCHW. ERMITTELT?\\

Die kinematischen und kinetischen Daten der Laufstrecke werden zur Berechnung der inversen Dynamik genutzt. Hierzu werden zunächst aus der kinematischen Untersuchung die Massenschwerpunkte der einzelnen Körperteile mit dem jeweils distalen (Index d) und proximalen (Index p) Gelenk und dem entsprechenden Gelenkkoeffizeinten $c_{joint}$ aus der antropomorphischen Tabelle (ANHANG A) gebildet:
\begin{equation}
x_{CoM} = (x_d - x_p) \cdot c_{joint} + x_p
\label{eq:CoM_x}
\end{equation}
\begin{equation}
y_{CoM} = (y_d - y_p) \cdot c_{joint} + y_p
\label{eq:CoM_y}
\end{equation}

Mit diesen Koordinaten werden die lineare Geschwindigkeit und Beschleunigung der Körperteile mittels Zentraldifferenz berechnet.\\
ZENTRALDIFFERENZ\\

Zur Berechnung der Winkelgeschwindigkeit und -beschleunigung sind folgende Gleichungen verwendet worden:\\
WINKELPROBLEM??\\
WIE GELÖST???\\

Alle Trajektorien, Geschwindigkeiten und Beschleunigungen wurden mittels gleitendem, gewichteten Mittelwert geglättet.\\
MITTELWERT!!!\\
DEN MÜSSEN WIR NOCH DARAUF ANWENDEN!!!\\
Für die Auswertung dir kinetischen Daten muss zunächst eine Waagenkallibration durchgeführt werden, um deren Drift zu bestimmen und die Spannungswerte mit dem Körpergewicht zu korrelieren. Das folgende Vorgehen wurde in allen drei Raumrichtungen durchgeführt: Aus einer Null-Messung ohne Belastung werden XXXX Werte gewählt und durch lineare Regression der Drift der Waage bestimmt. In einer zweiten Messung wird bei A, B, C und D kg jeweils 30~s gemessen. Nach Abzug des Drifts wird das Waagensignal für die vier Gweichte über 3000 Werte gemittelt und über diese vier Werte durch lineare Regression das Waagensignal mit Gewichten korreliert.\\
OFFSET-BERECHNUGN UND ABZUG BEI JEDER MESSUNG\\
HIER FEHLT NOCH DAS ZUSAMMENFASSEN DER KANÄLE\\


Aufbau siehe Kirltey et al
Die Rohdaten der X- und Y-Koordinaten und der Waagenmessungen werden für beide Versuche in Scilab ausgewertet. Die verwendetete Vorgehensweise wird daher hier für beide Versuche gebündelt erklärt, um den Arbeitsvorgang deutlich darzustellen.\\\\
X/Y-Koordinaten\\
Glättung\\
Gleitender Mittelwer\\
Plotten\\
Berechnung Winkel\\
Berechnung Beschl.\\
Berechnung Geschw.\\
\\

Inverse Kinematik\\
FELIX WAS HAST DU DA ALLES GEZAUBERT?!?!?


alle rohdaten an scilab gegeben für beide Versuch eund dann asugewertet\\
Gleichungen angeben\\
Daten filtern!\\
\textit{Gleichungen angeben}\\
\textbf{Kalibrierung}\\
- Nullmessung zur Bestimmung des Waagendrifts\\
- Kalibrierungsmessung\\
--> Bestimmung der der Abhängigkeit zwischen Belastungskraft und Messdaten\\
- Reduktion des Waagendrifts\\
- Regressionsgleichung von Messdaten und Belastungskraft bestimmen\\
\textbf{Datenaufnahme}\\
- Kräfte zusammenfassen\\
- Waagendrift aus Rohdaten rausrechnen\\
- Mittels der Kalibrierungsgleichungen die Rohdaten in Kräfte umrechnen\\
\textbf{Skalierung und Synchronisation!!}\\
- Bei der Skalierung wird die Datenrate der Videorate angepasst (hier also nur jeder zweite Datensatz). Gegebenenfalls muss zwischen den Datensätzen interpoliert werden.\\
- Bei der Datensynchronisation findet ein Abgleich des Videomaterials und der Bodenreaktionskräfte statt\\
\textbf{Digitalisierung des Videomaterials}\\
- Skalieren der Videoaufnahme (ACHTUNG! Referenzbild mit Maßstab erforderlich?!?!?)\\
- Tracken von allen Gelenken\\
- Segmentschwerpunkte berechnen (Fuß, Unter- und Oberschenkel)\\
\textbf{Datenberechnung\ KINEMATIK??}\\
- Berechnung von linearen Geschwindigkeiten und Beschleunigungen\\
- Berechnung von Winkelgeschwindigkeiten und Winkelbeschleunigungen\\
\textbf{Datenfilterung (gleitender Mittelwert)}\\
- ACHTUNG! Je nach Anzahl von Stützstellen und Iterationen müssen Bilder vor und nach dem Schrittzyklus in die Digitalisierung einbezogen werden. z.B. 3 Stützstellen und eine Iteration benötigt 1 Bild vorher und ein Bild nachher, um i-1 und n+1 zu berücksichtigen.\\
\textbf{Kinetische Berechnungen (Wagenzentrum rausrechnen?)}\\
Auf der Grundlage von David A. Winter werden:\\
- Kräfte und Momente in den Gelenken berechnet\\
- Berechnung mittels inverser Dynamik\\
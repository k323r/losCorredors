\documentclass{scrartcl}% siehe <http://www.komascript.de>
\usepackage{selinput}% Eingabecodierung automatisch ermitteln …
\SelectInputMappings{% … siehe <http://ctan.org/pkg/selinput>
  adieresis={ä},
  germandbls={ß},
}
\usepackage[ngerman]{babel}% Das Beispieldokument ist in Deutsch,
                % daher wird mit Hilfe des babel-Pakets
                % über Option ngerman auf deutsche Begriffe
                % und gleichzeitig Trennmuster nach den
                % aktuellen Rechtschreiberegeln umgeschaltet.
                % Alternativen und weitere Sprachen sind
                % verfügbar (siehe <http://ctan.org/pkg/babel>).
                
\usepackage[T1]{fontenc}
\usepackage{lmodern}
\begin{document}
% ----------------------------------------------------------------------------
% Titel (erst nach \begin{document}, damit babel bereits voll aktiv ist:
\titlehead{Kopf über dem Titel mit Leerstuhl u.\,ä.}% optional
\subject{Praktikumsbericht}% optional
\title{Ganganalyse des Menschen}% obligatorisch
\subtitle{Untertitel}% optional
\author{Vincent E. Focke}% obligatorisch
\date{Abgabe: 27.01.2017}% sinnvoll
\publishers{Betreuer: Nils Owsianowski \\ Leitung: Antonia B. Kesel}% optional
\maketitle% verwendet die zuvor gemachte Angaben zur Gestaltung eines Titels
% ----------------------------------------------------------------------------
% Inhaltsverzeichnis:
\tableofcontents
% ----------------------------------------------------------------------------
% Gliederung und Text:
\section{Einleitung}
\label{sec:motivation}
Dieser Abschnitt sollte sich mit der Aufgabenstellung befassen. Er kann auch
Grundlagen behandeln. Es kann jedoch sinnvoll sein, für die Grundlagen einen
heißt das, dass dies nicht geht SCheiße Fuß
eigenen Abschnitt zu verwenden.
\section{Durchführung}
\label{sec:durchfuehrung}
Hier erzählt man nun, was man alles gemacht hat.
\section{Schluss}
\label{sec:schluss}
Hierher gehört das Fazit und ggf. der Ausblick auf weitere Dinge, die getan
werden könnten.
\end{document}